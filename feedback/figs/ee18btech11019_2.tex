
\begin{circuitikz}[scale =2]
	
	\draw [thick,dash dot] (-4.3,-2.25) -- (-4.3,0.45) --(0.4,0.45) -- (0.4, -2.25);
	\draw [thick,dash dot] (-3.7,-2.25) -- (-3.7,-0.3) --(-0.3,-0.3) -- (-0.3, -2.25);
	\draw [thick,dash dot] (-4.3,-2.25) -- (-3.7,-2.25);
	\draw [thick,dash dot] (-0.3,-2.25) -- (0.4,-2.25);
	
	% Drawing a npn transistor
	\draw
	(-2,0.55)node[left]{Feedback $H(s)$}
	(0,0) node[coordinate](A){} 
	(0,-0.75)node[coordinate](B){$V_o$}
	(-4,0)node[coordinate](C){}
	(-4,-0.75)node[coordinate](D){}
	(-1,-0.75)node[coordinate](E){}
	% Making connections from transistor using relative coordinates
	% Labelling the transistor
	(A) to (B) to[L=$L_1$,v = $V_o$](0,-2) to node[ground]{}(0,-2)
    (C) to[C=c,i<^=$i_1$](0,0)
    (-4,-2) to[L=$L_2$](-4,-1) --(C)
    (-4,-2) to node[ground]{}(-4,-2)
    (D) --(-3,-0.75)
    (B) --(-2,-0.75)
    (E) to[R=R,*-*](-1,-2) to node[ground]{}(-1,-2)
    (-2,-0.75) to[cI =$g_mV_\pi$](-2,-2) to node[ground]{}(-2,-2)
    (-3,-2) to node[ground]{}(-3,-2)
    (-3,-0.8) to[open,v_=$V_\pi$](-3,-2.2)
	;
\end{circuitikz}
