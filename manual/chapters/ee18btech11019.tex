\begin{enumerate}[label=\thesubsection.\arabic*.,ref=\thesubsection.\theenumi]
\numberwithin{equation}{enumi}

\item  Consider the linear system :
\begin{align}
    \dot{x} = \myvec{-1 & 0\\0 & -2}x
\end{align}
with initial condition:  
x(0) = \myvec{1\\1}. Find x(t)


(A) x(t) =
\myvec{
e^{-t} & te^{-2t}\\
0 & e^{-2t}
}
\myvec{
1\\
1
}

\\

(B) x(t) =
\myvec{
e^{-t} & 0\\
0 & e^{2t}
}
\myvec{
1\\
1
}

\\

(C) x(t) =
\myvec{
e^{-t} & t^{2}e^{-2t}\\
0 & e^{-2t}
}
\myvec{
1\\
1
}

\\

(D) x(t) =
\myvec{
e^{-t} & 0\\
0 & e^{-2t}
}
\myvec{
1\\
1
}\\
\\
\solution Given expression is the state equation, as it can be written in the following form
\begin{align}
    \dot{x} =Ax + BU
\end{align} \\
Here, A =  \myvec{-1 & 0\\0 & -2} and U = 0\\
Solution to this can be given by
\begin{align}
    x(t) = \phi(t)x(0)
\end{align}
Where $\phi(t)$ is called the state-transition matrix and is given by\\
\begin{align}
    \phi(t) = \mathscr{L}^{-1}\myvec{\myvec{sI -A}^{-1}}
\end{align}\\
\begin{align}
    \myvec{sI-A} = \myvec{s+1 & 0\\ 0 & s+2}
\end{align}\\
\begin{align}
    \myvec{sI-A}^{-1} 
    = \myvec{\frac{1}{s+1} & 0\\
            0 & \frac{1}{s+2}}
\end{align}
\begin{align}
    \mathscr{L}^{-1}\myvec{\myvec{sI -A}^{-1}} = \myvec{
e^{-t} & 0\\
0 & e^{-2t}
}
\end{align}
Given x(0) = \myvec{1 \\ 1}
\begin{align}
    \therefore x(t) = \myvec{
e^{-t} & 0\\
0 & e^{-2t}
}
\myvec{
1\\
1
}\\ 
\therefore option (D)   
\end{align}
\newline
\newline

\emph{*Derivation for state transition matrix:}
\begin{align}
    \dot{x} =Ax
\end{align} 
Applying laplace transform, we get
\begin{align}
    sX(s) -x(0) = AX(s)\\
    \myvec{sI - A}X(s) = x(0)\\
    X(s) = \myvec{sI - A}^{-1}x(0)\\
\end{align} 
Taking inverse laplace transform
\begin{align}
    x(t) = \mathscr{L}^{-1}\myvec{\myvec{sI -A}^{-1}}x(0)
\end{align}
$\to$ Where $\mathscr{L}^{-1}$\myvec{\myvec{sI -A}^{-1}} is called State transition matrix.
\end{enumerate}
